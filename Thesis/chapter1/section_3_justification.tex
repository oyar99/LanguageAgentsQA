\section{Justification}
\label{sec:justification}

A robust agent-based system is essential for advancing state-of-the-art performance in natural language tasks, particularly QA and MHQA, which assess both language understanding and reasoning capabilities \cite{10.1561/1500000102}. These tasks are typically evaluated using benchmark datasets such as LoCoMo \cite{maharana-etal-2024-evaluating}, HotpotQA \cite{yang2018hotpotqa}, MuSiQue \cite{trivedi2021musique}, and 2WikiMultiHopQA (2Wiki) \cite{ho-etal-2020-constructing}.

\noindent Beyond academic benchmarks, agent-based architectures hold significant real-world potential. For instance, in specialized domains such as law, where documents are often lengthy and complex, they could enhance the effectiveness of QA systems \cite{regnlp-ws-2025-1}. They could also improve personal companion systems and psychological counseling applications, where maintaining continuity and context over time is critical for meaningful and supportive interactions \cite{Zhong_Guo_Gao_Ye_Wang_2024}. In the medical domain, where precision is essential and knowledge evolves rapidly, reliable QA systems remain critical. Unfortunately, recent studies indicate that current systems do not yet achieve the performance needed for such high-stakes applications \cite{ALONSO2024102938}.

\noindent Together, these motivations highlight the need for agent-based architectures that are adaptable, robust, and capable of supporting complex reasoning across diverse domains, particularly for QA tasks.