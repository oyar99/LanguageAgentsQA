\cleardoublepage

\chapter{Baseline Results}
\label{ch:results_apx}

This section presents detailed results for the baseline RAG systems evaluated in this work. Table \ref{tab:qa_results} reports question answering performance across different retrievers (BM25, ms-marco-bert-base-dot-v5, ColBERTv2), language models (GPT-4o-mini, o3-mini, QWen2.5-14B), and retrieval depths ($k \in \{5, 10, 20, 100\}$. These results serve as reference points for comparison with the language agent architectures.

\input{chapter4/qa_results}

\chapter{LLM Prompts}
\label{ch:prompts}

\section{Baselines}
\label{section:baselines_prompts}

This section presents the prompts used to implement the RAG baseline systems.

\begin{figure}[h]
    \centering
    \lstinputlisting{./appendix/files/prompt_base_qa.txt}
    \caption{Question Answering prompt used in MHQA benchmarks experiments with GPT-4o-mini and o3-mini.}
    \label{fig:qa-base}
\end{figure}
\begin{figure}[p]
    \centering
    \resizebox{0.9\textwidth}{!}{%
        \begin{minipage}{\textwidth}
            \lstinputlisting{./appendix/files/prompt_base_qa_qwen.txt}
        \end{minipage}%
    }
    \caption{Question Answering prompt used in MHQA benchmark experiments with Qwen2.5-14B-Instruct.}
    \label{fig:qa-base-qwen}
\end{figure}

\begin{figure}[h]
    \centering
    \lstinputlisting{./appendix/files/prompt_base_qa_all.txt}
    \caption{Question Answering prompt used in the FULL-CONTEXT setting for MHQA benchmarks.}
    \label{fig:qa-all}
\end{figure}
\begin{figure}[h]
    \centering
    \lstinputlisting{./appendix/files/prompt_base_locomo.txt}
    \caption{Question Answering prompt used in LoCoMo experiments.}
    \label{fig:locomo-base}
\end{figure}
\begin{figure}[h]
    \centering
    \lstinputlisting{./appendix/files/prompt_base_locomo_all.txt}
    \caption{Question Answering prompt used in the FULL-CONTEXT setting for LoCoMo.}
    \label{fig:qa-locomo-all}
\end{figure}

\cleardoublepage
\section{QA Agent}
\label{section:agent_prompts}

This section shows the prompts used to implement the variants of the QA Agent.

\begin{figure}[h]
    \centering
    \lstinputlisting{./appendix/files/prompt_qa_agent_reranking.txt}
    \caption{Prompt used in the QA Agent with Re-ranking second-stage retriever.}
    \label{fig:prompt_qa_agent_reranking}
\end{figure}
\begin{figure}[h]
    \centering
    \lstinputlisting{./appendix/files/prompt_qa_agent_reflection.txt}
    \caption{Prompt used in the QA agent with Self-Reflection to determine an alternative course of action.}
    \label{fig:prompt_qa_agent_reflection}
\end{figure}

\cleardoublepage
\section{DAG and Auto DAG Agent}
\label{section:dag_prompts}

This section presents the prompts used for implementing both the DAG Agent and the Auto DAG Agent. Figure \ref{fig:prompt_base_dag} shows the base prompt used to construct the graph $G$, excluding the representative few-shot examples provided to the model during our experiments.

\begin{figure}[h]
     \centering
    \lstinputlisting{./appendix/files/prompt_base_dag.txt}
    \caption{Base prompt used in both the DAG Agent and Auto DAG Agent to create a plan in form of a DAG.}
    \label{fig:prompt_base_dag}
\end{figure}

\noindent Figure \ref{fig:prompt_dag_execution} shows the prompt used to execute the DAG plan in the DAG Agent. Few-shot examples were also provided for each supported command during our experiments. In the actual implementation, the execution loop is controlled programmatically, so the command to execute (SOLVE, ALTERNATIVE\_ANSWER, or FINAL\_ANSWER) is always known in advance. Because the agent only needs to respond to a single, predetermined command at each step, we remove the unused command definitions and tool descriptions from the prompt to reduce token usage.

\begin{figure}[h]
    \centering
    \resizebox{0.9\textwidth}{!}{%
        \begin{minipage}{\textwidth}
\lstinputlisting{./appendix/files/prompt_dag_execution.txt}
        \end{minipage}%
    }
    \caption{Prompt used in the DAG Agent to execute the DAG plan.}
    \label{fig:prompt_dag_execution}
\end{figure}

\noindent Similarly, figure \ref{fig:prompt_auto_dag_execution} shows the prompt used to execute the DAG plan in the Auto DAG Agent. Few-shot examples were included in our experiments.

\begin{figure}[h]
     \centering
     \resizebox{0.8\textwidth}{!}{%
        \begin{minipage}{\textwidth}
\lstinputlisting{./appendix/files/prompt_auto_dag_execution.txt}
        \end{minipage}%
    }
    \caption{ReAct prompt used to execute the DAG plan in the Auto DAG Agent.}
    \label{fig:prompt_auto_dag_execution}
\end{figure}

\noindent Figure \ref{fig:prompt_auto_dag_synthesis} shows the prompt used to synthesize the final answer in the Auto DAG Agent.

\begin{figure}[h]
     \centering
    \lstinputlisting{./appendix/files/prompt_auto_dag_synthesis.txt}
    \caption{Prompt used in the Auto DAG Agent to synthesize the answer to the original question using the final state of the DAG $G$.}
    \label{fig:prompt_auto_dag_synthesis}
\end{figure}

\cleardoublepage
\section{Long-lived QA Agent}

Figure \ref{fig:prompt_cognitive} presents the prompt used to generate a correct reasoning chain after evaluating the correctness of an answer.

\begin{figure}[h]
     \centering
     \resizebox{0.7\textwidth}{!}{%
        \begin{minipage}{\textwidth}
\lstinputlisting{./appendix/files/prompt_cognitive.txt}
        \end{minipage}%
    }
    \caption{Prompt used to generate a reasoning chain given a correct answer and supporting documents.}
    \label{fig:prompt_cognitive}
\end{figure}

\cleardoublepage
\section{Evaluation}

This section demonstrates the prompt used to implement the LLM-based metric $L_1$.

\begin{figure}[h]
    \centering
    \lstinputlisting{./appendix/files/prompt_evaluation_judge.txt}
    \caption{Prompt used for evaluation for $L_1$ metric.}
    \label{fig:eval-judge}
\end{figure}
